\documentclass{cv_style}
\usepackage[letterpaper, total={7.6in, 10in}]{geometry}
\usepackage{parskip}
\usepackage{changepage}
\usepackage{booktabs}
\begin{document}
\name{\textbf{Matthew M. Lee}}
\begin{center}
    \contact{665 Huntington Avenue}{Boston}{MA}{mlee8@g.harvard.edu}{(916) 642-3497}   
\end{center}

\begin{adjustwidth}{0.7cm}{0.7cm} \parskip 8pt \begin{small} \textsc{Summary:} I am a PhD student in Population Health Sciences at the Harvard T.H. Chan School of Public Health Department of Nutrition. My research interests are focused on policies and interventions related to food and the environment that improve cardiometabolic health. Most recently, I have worked to evaluate the impact of sugar-sweetened beverage excise taxes on consumption patterns in the San Francisco Bay Area.
\end{small}
\end{adjustwidth}

\parskip -5pt 
\section{Education}
\parskip -5pt
\datedsubsection{\textbf{Harvard T.H. Chan School of Public Health}}{2019-}
\parskip -4pt \noindent PhD, Population Health Science; Concentration: Public Health Nutrition
\datedsubsection{\textbf{University of California, Berkeley School of Public Health}}{2017-2019}
\parskip -4pt M.S. Epidemiology \\
  \hangindent=0.7cm \rightskip=1cm \textsc{Thesis}: \textit{Maintenance of Reductions in Sugar-Sweetened Beverage Consumption, \newline Three Years After a Sugar-Sweetened Beverage Tax}
\parskip -8pt  
\datedsubsection{\textbf{University of California, Berkeley}}{2012-2016}
\parskip -4pt \noindent B.A. Public Health, \textsc{Minors:} Global Poverty and Practice, Music

\parskip -5pt 
\section{Teaching}
\datedsubsection{\textbf{Teaching Fellow, Population Health Sciences 2000}}{2020-2021}
\textit{Harvard Graduate School of Arts and Sciences}
\begin{itemize}
  \vspace{0em} \item Year-long core course for first-year PhD students in the Population Health Sciences doctoral program, covering concepts in sampling, estimation and statistical inference for regression models, model selection, survival and longitudinal analyses, measurement error, causal inference and mediation, and econometrics. 
  \item \parskip 1pt Led lab sessions and office hours with original lecture slides including applications using \texttt{R} programming language, development and grading of homework and exam assessments, and a primary role in course evaluation and refinement.  
\end{itemize}

\section{Work \& Research Experience}
\datedsubsection{\textbf{Research Assistant}}{2020-}
\textit{Harvard T.H. Chan School of Public Health, Department of Social and Behavioral Sciences} \\
The Childhood Obesity Intervention Cost-Effectiveness Study (CHOICES)
\begin{itemize}
    \item Supported cost-effectiveness analyses related to various policy interventions and long-run impacts on child obesity outcomes, including analysis of nationally representative data sources such as the National Health and Nutrition Examination Survey and the California Health Interview Survey, building and running microsimulation models, and validation of results against empirical data from.
\end{itemize}

\datedsubsection{\textbf{Graduate Researcher}}{2017-2019}
\textit{UC Berkeley School of Public Health Division of Community Health Sciences} \\
PI: Kristine Madsen, MD, MPH
\begin{itemize}
    \item Managed evaluation of sugar-sweetened beverage (SSB) taxes in Oakland, Berkeley, and San Francisco to assess the relationship between implementation and beverage consumption, using a quasi-experimental difference-in-differences design. In total for 2018, gathered information from 2800 individuals, 160 stores, and 60 store owners across 5 cities with ongoing work being done in Berkeley on retailer perceptions of the tax.
    \item \parskip 1pt Supported analysis for \textit{The Fit Study}, a three year cluster-randomized trial on the effect of BMI screening and reporting on child obesity outcomes. Examined the accuracy of parent perception of child weight status by treatment arm.
    \item \parskip 1pt Pulled and cleaned data from the California Department of Education, American Community Survey, and Decennial Census to merge with geospatial data for ancillary work examining the role of target marketing of SSB's and a project examining spillover effects of Physical Education related lawsuits on district PE policies in California.
    \item \parskip 1pt Managed and coordinated data collection and entry. Cleaned and analyzed data using Stata and R.
    \item \parskip 1pt Primary role in manuscript development, writing, and submission.
\end{itemize}

\datedsubsection{\textbf{Staff Research Associate}}{2016-2017}
\textit{UC Berkeley School of Public Health Division of Epidemiology} \\
PI: Mahasin Mujahid, PhD, MS; Patrick Bradshaw, PhD, MS
\begin{itemize}
    \item Managed components of research projects including literature review, Institutional Review Board (IRB) approval, accounting and finance reporting, data management, manuscript preparation and submission, and grant proposals.
    \item \parskip 1pt Teaching support for Public Health 150A, "Introduction to Epidemiology and Human Disease", including lecture content preparation, exam grading, and exam proctoring.
\end{itemize}

\datedsubsection{\textbf{Research Assistant}}{2015-2016}
\textit{UC Berkeley School of Public Health Division of Epidemiology} \\
PI: Mahasin Mujahid, PhD, MS
\begin{itemize}
    \item Completed detailed literature review for project examining the impact of community change initiatives on neighborhood collective efficacy; coded, transcribed, and analyzed qualitative data using Atlas.ti.
    \item \parskip 1pt Established codebook and developed theory of change models.
\end{itemize}

\datedsubsection{\textbf{Research Assistant}}{2014-2015}
\textit{UC San Francisco Pediatric Hematology \& Oncology} \\
PI: Biljana Horn, MD; Robert Goldsby, MD
\begin{itemize}
    \item Built comprehensive pediatric bone marrow transplant patient database in collaboration with Columbia University of risk factors associated with graft failure, including pre-transplant busulfan dose relation to chimerisms.
    \item \parskip 1pt Extracted patient medical records using EPIC EMR system and coordinated lab discussions and meetings.
\end{itemize}

\section{Publications}
\begin{enumerate}
    \item \textbf{Lee M}, Altman E, Madsen, KA. Secular trends in sugar-sweetened beverage consumption among adults, teens, and children: the California Health Interview Survey, 2011-2018. Preventing Chronic Disease. 2021. [In Press]
    \item \parskip 1pt Ponce J, Yuan H, Schillinger D, Mahmood H, \textbf{Lee MM}, Falbe J, Daniels R, Madsen KA. Retailer Perspectives on Sugar-Sweetened Beverage Taxes in the California Bay Area. Preventive Medicine Reports. 2020;101129.
    \item \parskip 1pt Falbe J, \textbf{Lee MM}, Kaplan S, Rojas NA, Ortega Hinojosa AM, Madsen KA. Higher Sugar-Sweetened Beverage Retail Prices After Excise Taxes in Oakland and San Francisco. Am J Public Health. 2020;e1–e7. 
    \item \parskip 1pt Mujahid MS, Sohn EK, Izenberg J, Gao X, Tulier M, \textbf{Lee MM}, Yen IH. Gentrification and Displacement in the San Francisco Bay Area: A Comparison of Measurement Approaches. Int J Environ Res Public Health [electronic article]. 2019;16(12).
    \item \parskip 1pt \textbf{Lee MM}, Falbe J, Schillinger D, Basu S, McCulloch C, Madsen K. Sugar-Sweetened Beverage Consumption 3 Years After the Berkeley, California, Sugar-Sweetened Beverage Tax. Am J Public Health. 2019;109(4):637–639. 
    \item \parskip 1pt \textbf{Lee MM}, Falbe J, Madsen KA. Secular Trends in Soda Consumption, California, 2011-2016. Prev Chronic Dis. 2019;16.

\end{enumerate}

\section{Service}

\datedsubsection{\textbf{Executive Director / Housing and Employment Coordinator}}{2013-2015}
\textit{The Suitcase Clinic}
\begin{itemize}
    \item Oversaw all volunteers, committees, and clinics aimed at providing health and social services to homeless and low-income populations in the Bay Area.
    \item \parskip 1pt Built and maintained relationships with partners including The City of Berkeley and Berkeley Free Clinic, worked effectively to secure \$40,000 in funding, and initiated process for attaining 501(c)3 status.
    \item \parskip 1pt Initiated data collection and electronic services database to bolster institutional memory, and created stable budget and accounting process.
    \item \parskip 1pt As Housing and Employment Coordinator, worked with clients to prepare applications for Section 8 housing, identify shelter resources, and reviewed resumes and application materials for job interviews. 
\end{itemize}


\datedsubsection{\textbf{Global Poverty and Practice Fellow}}{2015}
\textit{Makikita Quykuway}
\begin{itemize}
    \item Completed an internship at Makikita Quykuway, an NGO dedicated to alleviating health and resource disparities in peri-urban settlements in Peru, as a recipient of the Global Poverty and Practice Fellowship from the Blum Center for Developing Economies.
    \item \parskip 1pt Conducted non-profit program evaluation and developed health education materials, worked in health clinics and child labor reduction interventions.
\end{itemize}

\section{Skills}
\begin{itemize}
    \item Statistical Packages (R, Stata, SAS, Python)
    \item \parskip 1pt Geographic Information Systems (R)
    \item \parskip 1pt \LaTeX
    \item \parskip 1pt Qualitative analysis software (Atlas.ti)
    \item \parskip 1pt Version control (Google Drive Suite, Box, Git/Github)
    \item \parskip 1pt Qualtrics Survey Design
    \item \parskip 1pt Citation Software (Endnote, Refworks, Zotero)
    \item \parskip 1pt Microsoft Office (Word, PowerPoint, Excel)
    \item \parskip 1pt Adobe Creative Suite (Photoshop, Illustrator, InDesign, Premiere, Lightroom, Bridge)
\end{itemize}

\end{document}