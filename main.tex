\documentclass{cv_style}
\AtEveryBibitem{\clearfield{month}}
\AtEveryBibitem{\clearfield{day}}
\usepackage{enumitem}
\hypersetup{
  colorlinks = true,
  linkcolor = blue,
  urlcolor = blue,
  citecolor = blue
}
\begin{document}
\name{\textbf{Matthew M. Lee}}
\begin{center}
    \contact{665 Huntington Avenue}{Boston}{MA}{mlee8@g.harvard.edu}{\href{https://mmlee.me}{https://mmlee.me}}
\end{center}

%%%%%%%%%%%%%%%%%%%%%%%%%%%%%%%%%%%%%%%%%%
%% SUMMARY
%%%%%%%%%%%%%%%%%%%%%%%%%%%%%%%%%%%%%%%%%%

\begin{adjustwidth}{0.7cm}{0.7cm} \parskip 8pt \begin{small} \textbf{Summary:} I am a PhD candidate in Population Health Sciences at the Harvard T.H. Chan School of Public Health in the Department of Nutrition. My professional interests are focused on policies and interventions related to food systems and the environment that improve cardiometabolic outcomes, as well as the application of novel causal inference and simulation-based methods to advance population health and health equity research.  
\end{small}
\end{adjustwidth}

%%%%%%%%%%%%%%%%%%%%%%%%%%%%%%%%%%%%%%%%%%
%% EDUCATION
%%%%%%%%%%%%%%%%%%%%%%%%%%%%%%%%%%%%%%%%%%

\parskip -5pt 
\section{Education}
\parskip -5pt
% PhD Harvard
\datedsubsection{\textbf{Harvard T.H. Chan School of Public Health}}{Expected May 2024}
\parskip -4pt \noindent Doctor of Philosophy (PhD) in Population Health Science, Public Health Nutrition \\
\textbf{Thesis: Policy, Systems, and Environmental Levers to Improve Diet, Population Health, and Health Equity in the US}\\

\vspace{-1em}
    \begingroup
    \addtolength{\leftmargini}{3em} \begin{itemize}
        \item[\hspace{1em}Paper 1:] \textit{Estimating the cost-effectiveness of expanding universal free school lunch policies in the United States} \vspace{0.8em}
        \item [\hspace{1em}Paper 2:] \textit{Changes in adherence to the Dietary Guidelines for Americans and weight among adolescents in the Growing Up Today Study: A prospective cohort study} \vspace{0.8em} 
        \item [\hspace{1em}Paper 3:] \textit{Differences in diet following provision of lottery-based unconditional cash transfers during the COVID-19 pandemic in Chelsea, Massachusetts: A randomized lottery experiment}
    \end{itemize}
    \endgroup




% Berkeley MS
\datedsubsection{\textbf{University of California, Berkeley School of Public Health}}{2019}
\parskip -4pt Master of Science (MS) in Epidemiology \\
    \vspace{-1em}
    \begingroup
    \addtolength{\leftmargini}{3em} \begin{itemize}
        \item[\hspace{1em}Thesis:] \text{\textit{Sugar-Sweetened Beverage Consumption 3 Years After the Berkeley, California, Sugar-Sweetened Beverage Tax}}
    \end{itemize}
    \endgroup
\parskip -8pt  

% Berkeley BA
\datedsubsection{\textbf{University of California, Berkeley}}{2016}
\parskip -4pt \noindent  Bachelor of Arts (BA) in Public Health, \textsc{Minors:} Global Poverty, Music


%%%%%%%%%%%%%%%%%%%%%%%%%%%%%%%%%%%%%%%%%%
%% PUBLICATIONS
%%%%%%%%%%%%%%%%%%%%%%%%%%%%%%%%%%%%%%%%%%
\parskip -5pt 
\nocite{*}
\printbibliography[title=Publications]

\vspace{2em}

\textit{Reviewer for: PLOS ONE, BMC Public Health, Scientific Reports, Frontiers in Nutrition, Preventive Medicine Reports, Public Health, International Journal of Obesity, Journal of Nutritional Science}


\section{Presentations}
\begin{enumerate}[leftmargin = 2em]
    \item \textbf{Lee MM}, Kenney EL, Carlson K, Novick E, Portocarrero P, Rimm EB, Chen JT, Gortmaker SL, Stephenson BK, Liebman J. Impact of unconditional cash transfers on diet during the COVID-19 pandemic: A randomized lottery experiment. \textit{American Public Health Association Annual Meeting}. (2023). 
    \item \parskip 1pt \textbf{Lee MM}, Gibson LA, Hua SV, Lowery CM, Paul M, Roberto CA, Lawman HG, Bleich SN, Mitra N, Kenney EL. Changes in advertising and store stocking practices among small, independent beverage retailers following a sweetened beverage excise tax in Philadelphia: A difference-in-differences study. \textit{American Public Health Association Annual Meeting}. (2023).  
    \item \parskip 1pt Chapman L, Richardson S, Rimm EB, Gortmaker SL, \textbf{Lee MM}, Cohen J. Daily saturated fat and sodium content of school meals among a nationally representative sample of elementary schools in the United States. \textit{American Public Health Association Annual Meeting}. (2023). 
    \item \parskip 1pt \textbf{Lee MM}, Barrett J, Kenney EL, Gouck J, Cradock AL, Long MW, Ward ZJ, Rohrer B, Gortmaker SL. Impacts of a state-wide sugar-sweetened beverage excise tax in California: Projected benefits for population obesity and health equity. \textit{American Public Health Association Annual Meeting}. (2022).
    \item \parskip 1pt Kenney EL, \textbf{Lee MM}, Barrett J, Ward ZJ, Long MW, Cradock AL, Gortmaker SL. The cost-effectiveness and health equity impacts of improved nutrition standards in WIC, and the potential for maximizing WIC’s benefits for population health. \textit{American Public Health Association Annual Meeting}. (2022).
\end{enumerate}




%%%%%%%%%%%%%%%%%%%%%%%%%%%%%%%%%%%%%%%%%%
%% Honors
%%%%%%%%%%%%%%%%%%%%%%%%%%%%%%%%%%%%%%%%%%

\section{Awards \& Honors}

\textit{GSAS Professional Development Fund Recipient} \hfill {2023}\\
\textit{Irene M. & Fredrick J. Stare Nutrition Education Fund Awardee} \hfill {2022}\\
\textit{Berkowitz Fellowship in Public Health} \hfill {2021}\\
\textit{Certificate of Distinction in Teaching, Derek Bok Center for Teaching and Learning} \hfill {2021}\\
\textit{Certificate of Distinction in Teaching, Derek Bok Center for Teaching and Learning} \hfill {2020}\\
\textit{Simon, Arpi, and Marie Simonian Research Excellence in Nutrition Prize} \hfill {2020} \\
\textit{Prajna Chair's Scholarship in Public Health Nutrition} \hfill 2019 \\
\textit{Reshetko Family Scholarship} \hfill 2017

%%%%%%%%%%%%%%%%%%%%%%%%%%%%%%%%%%%%%%%%%%
%% WORK
%%%%%%%%%%%%%%%%%%%%%%%%%%%%%%%%%%%%%%%%%%

\section{Work \& Research Experience}

% Kenney
\datedsubsection{\textbf{Research Assistant}}{2020-Present}
\textit{Harvard T.H. Chan School of Public Health, Department of Nutrition} \\
PI: Erica Kenney, ScD, MPH
\begin{itemize}
    \item Led and supported data analysis, code review, and manuscript development and editing for projects related to nutrition policies aimed at addressing childhood obesity, improving food access, and reducing food insecurity -- including evaluations of US nutrition assistance programs, of the Philadelphia, PA, sweetened beverage excise tax, of racial/ethnic disparities in food-related advertising to young children and their families, and of WIC benefits expansions during the COVID-19 pandemic.
\end{itemize}

% CHOICES
\datedsubsection{\textbf{Research Assistant}}{2020-Present}
\textit{Harvard T.H. Chan School of Public Health, Department of Social and Behavioral Sciences} \\
PI: Steve Gortmaker, PhD, The Childhood Obesity Intervention Cost-Effectiveness Study (CHOICES)
\begin{itemize}
    \item Supported microsimulation-based cost-effectiveness analyses related to US policy interventions and their potential long-run impacts on child and adult outcomes, including health care spending, obesity, morbidity, and mortality. Analyzed dietary and health data from representative samples including the National Health and Nutrition Examination Survey, ran and synthesized output from simulation models, validated estimates against empirical data, and developed methods to streamline post-processing of results.
\end{itemize}

% UCB Madsen
\datedsubsection{\textbf{Graduate Researcher}}{2017-2019}
\textit{UC Berkeley School of Public Health Division of Community Health Sciences} \\
PI: Kristine Madsen, MD, MPH
\begin{itemize}
    \item Managed evaluation of sugar-sweetened beverage (SSB) taxes in Oakland, Berkeley, and San Francisco to assess the relationship between implementation and beverage consumption, using a quasi-experimental difference-in-differences design. Supported analysis for \textit{The Fit Study}, a three year cluster-randomized trial on the effects of BMI screening and reporting. Oversaw data collection, entry, analysis, and manuscript development and submission.
    \item \parskip 1pt Produced spatial data cross-linked with information from the California Department of Education, American Community Survey, and US Census for projects examining the role of targeted marketing of SSBs and the spillover effects of Physical Education related lawsuits on district PE policies in California.
\end{itemize}

% UCB Mujahid/Bradshaw
\datedsubsection{\textbf{Staff Research Associate}}{2016-2017}
\textit{UC Berkeley School of Public Health Division of Epidemiology} \\
PI: Mahasin Mujahid, PhD, MS; Patrick Bradshaw, PhD, MS
\begin{itemize}
    \item Managed research portfolio, including background literature reviews, Institutional Review Board (IRB) and ethics approval and documentation, accounting and finance reporting, data management, manuscript preparation and submission, and grant proposals. Provided teaching support for undergraduate-level Epidemiologic Methods course.
\end{itemize}

% UCB Mujahid/Carly
\datedsubsection{\textbf{Research Assistant}}{2015-2016}
\textit{UC Berkeley School of Public Health Division of Epidemiology} \\
PI: Mahasin Mujahid, PhD, MS
\begin{itemize}
    \item Completed detailed literature review on the impact of community change initiatives on neighborhood collective efficacy. Coded, transcribed, and analyzed qualitative data using Atlas.ti.
\end{itemize}

% UCSF Horn/Goldsby
\datedsubsection{\textbf{Research Assistant}}{2014-2015}
\textit{UC San Francisco Pediatric Hematology \& Oncology} \\
PI: Biljana Horn, MD; Robert Goldsby, MD
\begin{itemize}
    \item Built comprehensive pediatric bone marrow transplant patient database in collaboration with Columbia University of risk factors associated with graft failure, including pre-transplant busulfan dose relation to chimerisms.
    \item \parskip 1pt Extracted patient medical records using EPIC EMR system and coordinated lab discussions and meetings.
\end{itemize}

%%%%%%%%%%%%%%%%%%%%%%%%%%%%%%%%%%%%%%%%%%
%% TEACHING
%%%%%%%%%%%%%%%%%%%%%%%%%%%%%%%%%%%%%%%%%%

\parskip -5pt 
\section{Teaching}

% PHS 2000A/B
\datedsubsection{\textbf{Teaching Fellow, Population Health Sciences 2000 A/B}}{2020-2021, 2022}
``Quantitative Research Methods in Population Health Sciences'' | \textit{Harvard Graduate School of Arts and Sciences}
\begin{itemize}
  \vspace{0em} \item Year-long core course for first-year PhD students in the Population Health Sciences doctoral program, covering concepts in sampling, estimation and statistical inference for regression models, model selection, survival and longitudinal analyses, measurement error, causal inference and mediation, and econometrics. 
  \item \parskip 1pt Led lab sessions and weekly office hours with original lecture slides including applications using the R programming language, development and grading of homework and exam assessments, and primary role in course evaluation and refinement during transition to online/virtual learning. Lab lectures given included material on regression, measurement error, survival analysis, causal interaction and mediation, factor analysis, and meta-analysis.
\end{itemize}

% SBS 203/204
\datedsubsection{\textbf{Teaching Fellow, Social and Behavioral Sciences 203/204}}{2022, 2023}
``Program Implementation and Evaluation'' | \textit{Harvard T.H. Chan School of Public Health}
\begin{itemize}
  \vspace{0em} \item Semester course on conducting community health needs assessment, program planning, implementation, and evaluation, with a particular emphasis on community engagement and applied research. Required for masters students in Social and Behavioral Sciences. Foci includes health-related intervention for individuals, communities, organizations, and local/national groups and the various challenges that researchers and practitioners encounter when conducting this work "on the ground". 
  \item \parskip 1pt Facilitated in-class discussion on theoretical concepts and course materials, led lecture in cost effectiveness analysis and economic evaluation in the context of program planning, and provided critical feedback on student work, including a community health needs assessment and CDC \textit{Partnerships for Improving Community Health} grant application among others. 
\end{itemize}

% NUT 202
\datedsubsection{\textbf{Teaching Fellow, Nutrition 202}}{2022}
``The Biological Basis of Human Nutrition'' | \textit{Harvard T.H. Chan School of Public Health, Biological Sciences in Public Health}
\begin{itemize}
  \vspace{0em} \item Semester course on the biochemistry and metabolism of carbohydrates, fats, proteins, vitamins, and minerals in the context of human disease. Contemporary topics are emphasized. Particular emphasis is given to current knowledge of the mechanisms that may explain the role of diet in the causation and/or prevention of ischemic heart disease, diabetes, obesity, hypertension, and cancer. Required for masters and doctoral students in Nutrition.
  \item \parskip 1pt Coordinated weekly guest lectures and held weekly office hour review sessions on course material, led lecture in iron and copper metabolism in the context of human nutrition and health, drafted and evaluated exam questions and facilitated course feedback.
\end{itemize}

% PHS 2000A/B Tutor
\datedsubsection{\textbf{Tutor, Population Health Sciences 2000 A/B}}{2021-22}
``Quantitative Research Methods in Population Health Sciences'' | \textit{Harvard Graduate School of Arts and Sciences}
\begin{itemize}
  \vspace{0em} \item Hosted weekly tutoring sessions for 2-3 first-year PhD students in Population Health Sciences to support learning in PHS 2000A and B (course description above), review concepts in preparation for exams, go over solutions and concepts from problem sets, and provide mentoring on additional classes and resources.
\end{itemize}


%%%%%%%%%%%%%%%%%%%%%%%%%%%%%%%%%%%%%%%%%%
%% SERVICE
%%%%%%%%%%%%%%%%%%%%%%%%%%%%%%%%%%%%%%%%%%

\section{Service}

% Union
\datedsubsection{\textbf{Union Steward}}{2020-2021}
\textit{Harvard Graduate Student Union}
\begin{itemize}
    \item Supported department- and university-level organizing efforts to assess student needs, gauge contract priorities, and build social support amongst incoming and continuing doctoral students in the School of Public Health. 
\end{itemize}

% Suitcase Clinic
\datedsubsection{\textbf{Executive Director / Housing and Employment Coordinator}}{2013-2015}
\textit{The Suitcase Clinic}
\begin{itemize}
    \item Oversaw all volunteers, committees, and clinics aimed at providing health and social services to homeless and low-income populations in the Bay Area, including a course on issues on homelessness for UC Berkeley  students. 
    \item \parskip 1pt Built and maintained relationships with partners including The City of Berkeley and Berkeley Free Clinic, worked effectively to secure \$40,000 in funding, and initiated process for attaining 501(c)3 status. Initiated data collection and electronic services database to bolster institutional memory, and created stable budget and accounting process.
\end{itemize}

% GPP Fellow
\datedsubsection{\textbf{Global Poverty and Practice Fellow}}{2015}
\textit{Makikita Quykuway}
\begin{itemize}
    \item Completed internship at Makikita Quykuway, a non-governmental organization dedicated to alleviating health and resource disparities in peri-urban informal settlements in Peru, as a recipient of the Global Poverty and Practice Fellowship from the Blum Center for Developing Economies at UC Berkeley.
    \item \parskip 1pt Conducted non-profit program evaluation and developed health education materials, worked in health clinics and supported child labor reduction interventions.
\end{itemize}

%%%%%%%%%%%%%%%%%%%%%%%%%%%%%%%%%%%%%%%%%%
%% SKILLS
%%%%%%%%%%%%%%%%%%%%%%%%%%%%%%%%%%%%%%%%%%

\section{Skills}
\begin{itemize}
    \item Programming languages: 
    \begin{itemize}
        \item Statistical analysis (R, Stata, SAS, Python, Stan)
        \item \parskip 1pt Geographic Information Systems (R)
        \item \parskip 1pt Typesetting (\LaTeX, R Markdown, R Sweave, Markdown) 
        \item \parskip 1pt Presentations (react.js, R) 
    \end{itemize}
    \item \parskip 1pt Software: 
    \begin{itemize}
        \item \parskip 1pt Qualitative analysis (Atlas.ti)
        \item \parskip 1pt Reference Management (Endnote, Refworks, Zotero, Mendeley)
        \item \parskip 1pt Microsoft Office/Google Suite (Microsoft Word, PowerPoint, Excel; Google Docs, Sheets, Slides)
        \item \parskip 1pt Adobe Creative Suite (Photoshop, Illustrator, InDesign, Premiere, Lightroom, Bridge)
    \end{itemize}
    \item \parskip 1pt Version control (Git/GitHub)
    \item \parskip 1pt Survey design (Qualtrics, Google Forms)
\end{itemize}

\end{document}